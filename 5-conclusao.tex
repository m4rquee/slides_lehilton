\begin{frame}{Outros Resultados}
    \begin{thm}[\cite{Cygan15}]
        Existem algoritmos FPT (parametrização natural) para os seguintes problemas em grafos planares:

        \begin{enumerate}[]
            \item Isomorfismo de Subgrafos;
            \item Bisseção Mínima.
        \end{enumerate}
    \end{thm}
\end{frame}

\begin{frame}{Resultado Geral}
    A única propriedade dos grafos planares que utilizamos é serem fechados por menores e terem \textbf{\emph{largura arbórea local}} limitada.
    \pause \bigbreak
    Uma classe de grafos $\mathcal{G}$ tem \textbf{\emph{largura arbórea local}} limitada se, para todo $G \in \mathcal{G}$ e $v \in V(G)$, vale $tw(G_v^r) \le f(r)$, onde $G_v^r$ é o subgrafo induzido pelos vértices a distância até $r$ de $v$.
\end{frame}

\begin{frame}{Limitações}
    \centering
    A técnica de \emph{Shifting}/Baker é eficaz para problemas ``locais''~\dots
    \bigbreak\pause
    \dots~mas falha em casos como TSP ou Árvore de Steiner.
\end{frame}

\begin{frame}{Generalização}
    \centering\large
    \textbf{\emph{Decomposição por contração}} em grafos livres de $H$:
    \pause
    \bigbreak
    \begin{thm}[\cite{Dem11}]
        Para todo grafo fixo $H$, existe uma constante $c_H$ tal que, \pause
        para todo inteiro $k \geq 1$, qualquer grafo livre de $H$-menores pode ter suas arestas particionadas em $k + 1$ classes de cor, \pause
        de modo que a contração de qualquer uma dessas classes resulta em um grafo com largura arbórea no máximo $c_H \cdot k$.
        \pause
        \bigbreak
        Além disso, tal partição pode ser encontrada em tempo polinomial.
    \end{thm}
\end{frame}